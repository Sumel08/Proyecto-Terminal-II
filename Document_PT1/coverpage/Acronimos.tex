
\chapter*{Acr�nimos}

Hagan una lista de los acr�nimos empleados en su trabajo.

Por si no saben qu� es un acr�nimo les dejo lo siguiente:\\
\\
un acr�nimo (del griego \textgreek{>'akroc} \LyXbar{}transliterado
como akros\LyXbar{} \textquoteleft{}extremo\textquoteright{}, y \textgreek{>'onoma}
\LyXbar{}trasliterada como �noma\LyXbar{} \textquoteleft{}nombre\textquoteright{})
puede ser una sigla que se pronuncia como una palabra \LyXbar{}y que
por el uso acaba por incorporarse al l�xico habitual en la mayor�a
de casos,1 como l�ser (Light Amplification by Stimulated Emission
of Radiation)\LyXbar{} o tambi�n puede ser un vocablo formado al unir
parte de dos palabras. Este �ltimo tipo de acr�nimos funden dos elementos
l�xicos tomando, casi siempre, del primer elemento el inicio y del
segundo el final, como bit (Binary digit).
