
\section{Planteamiento del problema}

La Organización Mundial de la Salud (OMS) estima que hay más de 300 millones de personas con discapacidad auditiva en el mundo. Y de acuerdo al XIII Censo General de Población y Vivienda 2010, existen un total de 498 mil 640 personas sordas en México, donde 273 mil 216 son hombres y 225 mil 424 son mujeres \cite{Valencia2012}. La mayoría de estas personas vive en condiciones precarias, lo cual acentúa la falta de acceso a la educación superior. La Dirección de Educación Especial de la Secretaría de Educación Pública (SEP) está a cargo de instruir a las personas sordas, sin embargo, sólo se oferta hasta el nivel básico de secundaria por lo que al egresar estas personas no tiene opción para continuar \cite{Miranda2003}.

	Otro problema presente en esta población es que no existe  vocabulario científico ni técnico para explicar ciertos procesos a nivel preparatoria, además también se enfrentan al problema que se relaciona con las nociones de lengua escrita que reciben en el aula, pues no les son suficientes para leer y escribir de forma adecuada, lo cual obstaculiza su ingreso al nivel medio superior \cite{Miranda2003}.	

	Debido a la falta de interés de las personas en aprender el lenguaje de señas las personas que sufren esta deficiencia del habla quedan excluidas de la sociedad en general, pues sólo pueden establecer comunicación con grupos especiales que dominan este sistema de señas.

	Estas personas quedan excluidas debido a que no logran una adaptación completa en la sociedad pues no pueden comunicarse con los demás, expresarse o hacer que los demás los escuchen, y una solución podría ser que todos aprendamos este lenguaje, el problema es la falta de educación orientada a este aspecto y el desinterés en esta lengua. 


	Por otro lado, las personas que pierden la audición pueden tener algunos problemas sociales como:

	\begin{itemize}
		\item Aislamiento y retraimiento.
		\item Pérdida de atención.
		\item Falta de concentración.
		\item Problemas en el trabajo.
		\item Problemas de comunicación.
	\end{itemize}


	Por lo que se plantea la pregunta ¿Cómo realizar una aplicación que utilice el reconocimiento de voz para facilitar la comunicación entre personas que no utilizan el lenguaje de señas con personas que sí?